\documentclass{beamer}
\mode<presentation> {
\usetheme[compactlogo, white]{Wisconsin}
}

\usepackage{graphicx} % Allows including images
\usepackage{booktabs} % Allows the use of \toprule, \midrule and \bottomrule in tables


%----------------------------------------------------------------------------------------
%   TITLE PAGE
%----------------------------------------------------------------------------------------

\title[Project 1 - Measurement]{Use Graph to Talk About Measurement of Different Communication Abstractions in Distributed System} % The short title appears at the bottom of every slide, the full title is only on the title page

\author{Kan Wu} % Your name
\institute[UW-Madison] % Your institution as it will appear on the bottom of every slide, may be shorthand to save space
{
University of Wisconsin, Madison \\ % Your institution for the title page
\medskip
\textit{kanwu@cs.wisc.edu} % Your email address
}
\date{\today} % Date, can be changed to a custom date


\begin{document}
% Few starting pages 
    \begin{frame}
        \titlepage % Print the title page as the first slide
    \end{frame}

    \begin{frame}
        \frametitle{Overview} % Table of contents slide, comment this block out to remove it
        \tableofcontents % Throughout your presentation, if you choose to use \section{} and \subsection{} commands, these will automatically be printed on this slide as an overview of your presentation
    \end{frame}


% Begin each section 
\section{Timing}
    \subsection{Record your timing during loop and simple statements}
    \begin{frame}
        \frametitle{Record your timing during loop and simple statements}
        It proves clock\_gettime is precise enough to catch time difference even between 2 or 3 mov like assembly code.

    \end{frame}


   
\section{Using My Timer}
    \subsection{Verify Jeff Dean's numbers}
    \begin{frame}
        \frametitle{Verify Jeff Dean's numbers}
        Use our function or other methods to measure some numbers in Jeff Dean's paper. 

        
    \end{frame}

\section{Reliable Communications}
    \subsection{Interfaces and Reliable Design}
    \begin{frame}
        \frametitle{Interfaces and Reliable Design}
        This part is to overview my measurement of our communication library.

        

    \end{frame}
    \subsection{Performance and Reliability}
    \begin{frame}
        \frametitle{Reliable Communications: Performance}
        Overhead and RTT. 

        \begin{itemize}
            \item Overhead of sending  
            \item RRT
            \item Single of Two machines?
            \item Bandwitdh and analysis 

        \end{itemize}
    \end{frame}
    \begin{frame}
        \frametitle{Reliable Communications: Performance}
        Bandwidth. 

        What limits my bandwidth: 

    \end{frame}
    
    \begin{frame}
        \frametitle{Reliable Communications: Reliability}
        This part is to overview my measurement of our communication library.

        Compiler didn't influence my code large. 

  
    \end{frame}



\section{Google RPC and Apache Thrift}
    \begin{frame}
        \frametitle{Marshalling a Message}
        Method: SerializetoString, SendMessageBegin ... 
     
    \end{frame}
    \begin{frame}
        \frametitle{RTT Measuring}
        Echo Server

     
    \end{frame}
    \begin{frame}
        \frametitle{BindWidth Measuring}
        

     
    \end{frame}
    \begin{frame}
        \frametitle{Compiler Difference}
        

     
    \end{frame}
    

\end{document}